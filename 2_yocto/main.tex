\documentclass{beamer}

\mode<presentation>
{
    \usetheme{Warsaw}
    \usecolortheme{seahorse}
    \setbeamercovered{transparent}
}

\usepackage[english]{babel}
\usepackage{times}
\usepackage[utf8]{inputenc}
\usepackage[T1]{fontenc}
\usepackage{graphicx}
\usepackage{listings}
\usepackage{textcomp}

\lstdefinestyle{Console}{
    keepspaces=true,
    basicstyle=\scriptsize\ttfamily,
    numberstyle=\scriptsize,
    numbers=none,
    frame=tblr,
    columns=fullflexible,
    backgroundcolor=\color{blue!10},
    linewidth=1\linewidth,
    xleftmargin=0\linewidth,
    moredelim=[is][\textbf]{@@@}{@@@}
}

\lstdefinestyle{TinyConsole}{
    keepspaces=true,
    basicstyle=\tiny\ttfamily,
    numberstyle=\tiny,
    numbers=none,
    frame=tblr,
    columns=fullflexible,
    backgroundcolor=\color{blue!10},
    linewidth=1\linewidth,
    xleftmargin=0\linewidth,
    moredelim=[is][\textbf]{@@@}{@@@}
}

\title[Yocto: The Practical Guide]{Yocto}
\subtitle{The Practical Guide}
\author[M. Jarycki, K. Laskowski]{
    Marek Jarycki \\
    Krzysztof Laskowski
}

\begin{document}

\begin{frame}
    \titlepage
\end{frame}

\begin{frame}
    \tableofcontents
\end{frame}

\section{Distro vs ECL}

\subsection{Fundamentals}

\begin{frame}{what is das Linux}
\end{frame}

\begin{frame}{what is distro}
\end{frame}

\begin{frame}{what is Linux distro}
\end{frame}

\begin{frame}{what is version}
\end{frame}

\begin{frame}{what is semantic version}
\end{frame}

\begin{frame}{what is Linux distro version}
\end{frame}

\begin{frame}{what is Linux distro package metadata}
\end{frame}

\subsection{The world of ECL}

\begin{frame}{what is ECL}
\end{frame}

\begin{frame}{what is LFS}
\end{frame}

\begin{frame}{what is relation between distro version and ecl}
\end{frame}

\section{Artifacts}

\subsection{SDK}

\begin{frame}{What is SDK?}
    \begin{block}{}
        An \textbf{SDK} is a toolchain and sysroot.
    \end{block}
    \begin{block}{}
        It has essentials needed to develop for particular platform.
    \end{block}
    \begin{block}{}
        It should work on any machine developer may use.
    \end{block}
    \begin{block}{}
        It is self contained.
    \end{block}
\end{frame}

\begin{frame}[fragile]{SDK installs}
\begin{block}{}
SDK comes in form of installable shell script: shar
(shell archive).
\end{block}
\begin{lstlisting}[style=Console]
$ ./***-toolchain-***.sh
Enter target directory for SDK (default: /opt/***/1.6.1): sdk
You are about to install the SDK to "/***/sdk". Proceed[Y/n]?
Extracting SDK...done
Setting it up...done
SDK has been successfully set up and is ready to be used.
\end{lstlisting}
\end{frame}

\begin{frame}[fragile]{SDK has a directory structure}
\begin{block}{}
\begin{itemize}
\item{setup script}
\item{target sysroot}
\item{toolchain sysroot}
\end{itemize}
\end{block}
\begin{lstlisting}[style=TinyConsole]
sdk
|-- environment-setup-core2-64-nokia-linux
|-- site-config-core2-64-nokia-linux
|-- sysroots
|   |-- core2-64-nokia-linux
|   |   |-- etc
|   |   |-- lib64
|   |   |-- usr
|   |   \-- var
|   \-- i686-nsnsdk-linux
|       |-- bin
|       |-- etc
|       |-- lib
|       |-- usr
|       |-- var
|       \-- x86_64-pc-linux-gnu
\-- version-core2-64-nokia-linux
\end{lstlisting}
\end{frame}

\begin{frame}[fragile, t]{SDK has environment script}
    \begin{block}{}
        Sets path, auxiliary and program specific variables:
        \begin{itemize}
            \scriptsize
            \item<1->{to know location of sysroots}
            \item<2->{to use \verb|[archprefix]-gcc|, \verb|cmake|, \verb|opkg|}
            \item<3->{\verb|CC|, \verb|CXX|, \verb|LD| for build tool to detect}
            \item<4->{for \verb|pkg-config|, \verb|opkg|, \verb|python| to use sdk}
            \item<5->{libdir and dynlinker to execute binaries}
        \end{itemize}
    \end{block}
    \begin{onlyenv}<1>
\begin{lstlisting}[style=TinyConsole]
export SDKTARGETSYSROOT=/***/sdk/sysroots/core2-64-nokia-linux
export OECORE_NATIVE_SYSROOT="/***/sdk/sysroots/i686-nsnsdk-linux"
\end{lstlisting}
    \end{onlyenv}
    \begin{onlyenv}<2>
\begin{lstlisting}[style=TinyConsole]
export PATH=/***/sdk/sysroots/i686-nsnsdk-linux/usr/bin: \
            /***/sdk/sysroots/i686-nsnsdk-linux/usr/bin/x86_64-nokia-linux: \
            $PATH
\end{lstlisting}
    \end{onlyenv}
    \begin{onlyenv}<3>
\begin{lstlisting}[style=TinyConsole]
export CC="x86_64-pc-linux-gnu-gcc -m64 --sysroot=$SDKTARGETSYSROOT"
export CXX="x86_64-pc-linux-gnu-g++ -m64 --sysroot=$SDKTARGETSYSROOT"
export CPP="x86_64-pc-linux-gnu-gcc -E -m64 --sysroot=$SDKTARGETSYSROOT"
export AS="x86_64-pc-linux-gnu-as"
export LD="x86_64-pc-linux-gnu-ld --sysroot=$SDKTARGETSYSROOT"
\end{lstlisting}
    \end{onlyenv}
    \begin{onlyenv}<4>
\begin{lstlisting}[style=TinyConsole]
export PKG_CONFIG_SYSROOT_DIR=$SDKTARGETSYSROOT
export PKG_CONFIG_PATH=$SDKTARGETSYSROOT/usr/lib64/pkgconfig
export OFFLINE_ROOT=$SDKTARGETSYSROOT
export PYTHONPATH=$OECORE_NATIVE_SYSROOT/usr/lib/python2.7/site-packages:$PYTHONPATH
\end{lstlisting}
    \end{onlyenv}
    \begin{onlyenv}<5>
\begin{lstlisting}[style=TinyConsole]
export SYSROOT_LIB_DIR=$SDKTARGETSYSROOT/usr/lib64:$SDKTARGETSYSROOT/lib64
export SYSROOT_DYN_LINKER=$SDKTARGETSYSROOT/lib64/ld-linux-x86-64.so.2
\end{lstlisting}
    \end{onlyenv}
\end{frame}

\begin{frame}[fragile]{SDK toolchain has compilers and tools}
\begin{lstlisting}[style=Console]
bin
|-- x86_64-pc-linux-gnu-g++
|-- x86_64-pc-linux-gnu-gcc
|-- x86_64-pc-linux-gnu-ld
|-- x86_64-pc-linux-gnu-nm
|-- x86_64-pc-linux-gnu-readelf
|-- x86_64-pc-linux-gnu-size
|-- x86_64-pc-linux-gnu-strings
\-- x86_64-pc-linux-gnu-strip
\end{lstlisting}
\begin{lstlisting}[style=Console]
usr
\-- bin
    |-- cmake
    |-- opkg -> opkg-cl
    |-- opkg-cl
    |-- pkg-config
    |-- prophyc
    \-- protoc
\end{lstlisting}
\end{frame}

\begin{frame}[fragile, t]{SDK sysroot holds target-related stuff}
    \begin{block}{}
        \begin{itemize}
            \item<1->{executables \& scripts}
            \item<2->{libraries \& headers \& pc files}
            \item<3->{configuration}
            \item<4->{data}
        \end{itemize}
    \end{block}
    \begin{onlyenv}<1>
\begin{lstlisting}[style=Console]
usr
\-- bin
    \-- CCSDaemonExe
\end{lstlisting}
    \end{onlyenv}
    \begin{onlyenv}<2>
\begin{lstlisting}[style=Console]
usr
|-- include
|   \-- boost
|       \-- chrono
|           \-- chrono.hpp
\-- lib64
    |-- pkgconfig
    |   \-- prophy.pc
    |-- libboost_chrono.so -> libboost_chrono.so.1.59.0
    \-- libboost_chrono.so.1.59.0
\end{lstlisting}
    \end{onlyenv}
    \begin{onlyenv}<3>
\begin{lstlisting}[style=Console]
etc
\-- opkg
    \-- opkg.conf
\end{lstlisting}
    \end{onlyenv}
    \begin{onlyenv}<4>
\begin{lstlisting}[style=Console]
usr
\-- share
    \-- bpf
        \-- bpf.xml
\end{lstlisting}
    \end{onlyenv}
\end{frame}

\begin{frame}{SDK sysroot is filled bottom-up}
    \begin{block}{}
        \begin{itemize}
            \item{may be empty after sdk deployment}
            \item{will be filled according to dependencies}
            \begin{itemize}
                \item{by package manager \& server}
                \item{by manual package installation}
                \item{by manual build \& installation from source}
            \end{itemize}
        \end{itemize}
    \end{block}
\end{frame}

\subsection{Package}

\begin{frame}[fragile]{Package has format}
    \begin{block}{}
        Packages fulfil a certain format, which package manager can understand.
    \end{block}
    \begin{block}{}
        Opkg requires \verb|.ipk| packages, format similar to \verb|.deb|.
    \end{block}
\end{frame}

\begin{frame}[fragile]{Package has name}
    \begin{block}{}
        There are 3 sections separated by underscores
        \begin{itemize}
            \item<2->{name \only<3->{and purpose}}
            \item<4->{version}
            \item<5->{architecture}
        \end{itemize}
    \end{block}
\begin{onlyenv}<1>\begin{lstlisting}[style=Console]
liblim-dev_0~FL00-LIM-0001-07-2898-r0_core2-64.ipk
\end{lstlisting}\end{onlyenv}
\begin{onlyenv}<2>\begin{lstlisting}[style=Console]
lib@@@lim@@@-dev_0~FL00-LIM-0001-07-2898-r0_core2-64.ipk
\end{lstlisting}\end{onlyenv}
\begin{onlyenv}<3>\begin{lstlisting}[style=Console]
@@@liblim-dev@@@_0~FL00-LIM-0001-07-2898-r0_core2-64.ipk
\end{lstlisting}\end{onlyenv}
\begin{onlyenv}<4>\begin{lstlisting}[style=Console]
liblim-dev_@@@0~FL00-LIM-0001-07-2898-r0@@@_core2-64.ipk
\end{lstlisting}\end{onlyenv}
\begin{onlyenv}<5>\begin{lstlisting}[style=Console]
liblim-dev_0~FL00-LIM-0001-07-2898-r0_@@@core2-64@@@.ipk
\end{lstlisting}\end{onlyenv}
\end{frame}

\begin{frame}[fragile]{Package has purpose}
    \begin{block}{}
        Usually single project is partitioned to several packages.
        Each package serves a different purpose.
    \end{block}
    \begin{itemize}
        \item{
            dynamic library \hfill to run
\begin{lstlisting}[style=Console]
liblim1_0~FL00-LIM-0001-07-2898-r0_core2-64.ipk
\end{lstlisting}
        }
        \item{
            debug symbols and source code \hfill to debug
\begin{lstlisting}[style=Console]
liblim-dbg_0~FL00-LIM-0001-07-2898-r0_core2-64.ipk
\end{lstlisting}
        }
        \item{
            headers, library link, \verb|.pc| file \hfill to link against
\begin{lstlisting}[style=Console]
liblim-dev_0~FL00-LIM-0001-07-2898-r0_core2-64.ipk
\end{lstlisting}
        }
        \item{
            static library \hfill to link against all symbols
\begin{lstlisting}[style=Console]
liblim-staticdev_0~FL00-LIM-0001-07-2898-r0_core2-64.ipk
\end{lstlisting}
        }
    \end{itemize}
\end{frame}

\begin{frame}[fragile]{Package has directory layout}
    \begin{block}{}
        Package is a compressed archive with directory structure
        \begin{itemize}
            \item{piece of sysroot to install}
            \item{metadata}
        \end{itemize}
    \end{block}
\begin{lstlisting}[style=Console]
liblim-dev_0~FL00-LIM-0001-07-2898-r0_core2-64.ipk
|-- CONTENTS
|   \-- usr
|       |-- include
|       |   \-- lim
|       |       \-- ...
|       \-- lib64
|           |-- liblim.so -> liblim.so.1
|           \-- pkgconfig
|               \-- lim.pc
|-- DEBIAN
|   \-- control
|-- INFO
\-- INSTALL
\end{lstlisting}
\end{frame}

\begin{frame}[fragile]{Package has metadata}
    \begin{block}{}
        File \verb|DEBIAN/control| contains package metadata.
    \end{block}
\begin{lstlisting}[style=Console]
@@@Package: liblim-dev@@@
@@@Version: 0~FL00-LIM-0001-07-2898-r0@@@
Description: *****
Section: devel
Priority: optional
Maintainer: *****
License: CLOSED
@@@Architecture: core2-64@@@
OE: lim
@@@Depends: liblim1 (= 0~FL00-LIM-0001-07-2898-r0)@@@
Recommends: *****
Source: *****
\end{lstlisting}
\end{frame}

\begin{frame}[fragile]{Package has dependencies}
    \begin{block}{}
        Packages usually depend on others.
    \end{block}
\begin{lstlisting}[style=Console]
Package: zeya
...
Depends: python (>= 2.7.1-0ubuntu2), vorbis-tools,
    python-simplejson, python-tagpy
Recommends: mpg123, flac, faad
\end{lstlisting}
    \begin{block}{}
        This allows package manager to fetch dependencies
        automatically. It helps to keep system integrity
        during upgrades or removals.
    \end{block}
\end{frame}

\begin{frame}[fragile]{Package has project information}
    \begin{block}{}
        Package metadata alone can point to
        rich source of project related information.
    \end{block}
\begin{lstlisting}[style=Console]
Package: opera
...
Maintainer: Opera Packaging Team <packager@opera.com>
Bugs: https://bugs.opera.com/wizard/
Homepage: http://www.opera.com/browser/
\end{lstlisting}
    \begin{block}{}
        This data is easy to keep up to date since
        packages update constantly.
    \end{block}
\end{frame}

\begin{frame}[fragile]{Package can have documentation}
    \begin{block}{}
        \renewcommand{\arraystretch}{1.5}
        \begin{tabular}{l l}
            readme & \small\verb|/usr/share/doc/valgrind/README| \\
            changelog & \small\verb|/usr/share/doc/valgrind/NEWS.Debian.gz| \\
            authors & \small\verb|/usr/share/doc/valgrind/AUTHORS| \\
            man pages & \small\verb|/usr/share/man/man1/valgrind.1.gz| \\
            html docs & \small\verb|/usr/share/doc/valgrind/html/index.html| \\
        \end{tabular}
    \end{block}
\end{frame}

\begin{frame}[fragile]{Package can be arch-agnostic}
    \begin{block}{}
        Packages which contain no machine-specific files,
        e.g. only data or headers, can have architecture \verb|all|.
    \end{block}
    \begin{block}{}
        Such packages are reused
        between different architectures of a distro.
    \end{block}
\end{frame}

\subsection{Index}

\begin{frame}
\end{frame}

\section{Roles}

\subsection{Maintainer}

\begin{frame}
\end{frame}

\subsection{Developer}

\begin{frame}
\end{frame}

\end{document}
