\documentclass{beamer}

\mode<presentation>
{
    \usetheme{Warsaw}
    \usecolortheme{seahorse}
    \setbeamercovered{transparent}
}

\usepackage[english]{babel}
\usepackage{times}
\usepackage[utf8]{inputenc}
\usepackage[T1]{fontenc}
\usepackage{graphicx}
\usepackage{listings}
\usepackage{textcomp}

\lstdefinestyle{Console}{
    keepspaces=true,
    basicstyle=\scriptsize\ttfamily,
    numberstyle=\scriptsize,
    numbers=none,
    frame=tblr,
    columns=fullflexible,
    backgroundcolor=\color{blue!10},
    linewidth=1\linewidth,
    xleftmargin=0\linewidth
}

\lstdefinestyle{TinyConsole}{
    keepspaces=true,
    basicstyle=\tiny\ttfamily,
    numberstyle=\tiny,
    numbers=none,
    frame=tblr,
    columns=fullflexible,
    backgroundcolor=\color{blue!10},
    linewidth=1\linewidth,
    xleftmargin=0\linewidth
}

\title[Yocto: The Practical Guide]{Yocto}
\subtitle{The Practical Guide}
\author[M. Jarycki, K. Laskowski]{
    Marek Jarycki \\
    Krzysztof Laskowski
}

\begin{document}

\begin{frame}
    \titlepage
\end{frame}

\begin{frame}
    \tableofcontents
\end{frame}

\section{Distro vs ECL}

\subsection{Fundamentals}

\begin{frame}
    what is das Linux
\end{frame}

\begin{frame}
    what is distro
\end{frame}

\begin{frame}
    what is Linux distro
\end{frame}

\begin{frame}
    what is version
\end{frame}

\begin{frame}
    what is semantic version
\end{frame}

\begin{frame}
    what is Linux distro version
\end{frame}

\begin{frame}
    what is Linux distro package metadata
\end{frame}

\subsection{The world of ECL}

\begin{frame}
    what is ECL
\end{frame}

\begin{frame}
    what is LFS
\end{frame}

\begin{frame}
    what is relation between distro version and ecl
\end{frame}

\section{Artifacts}

\subsection{SDK}

\begin{frame}{What is SDK?}
    \begin{block}{}
        An \textbf{SDK} is a toolchain and sysroot.
    \end{block}
    \begin{block}{}
        It has essentials needed to develop/crosscompile for particular platform.
    \end{block}
\end{frame}

\begin{frame}[fragile]{SDK installs}
\begin{block}{}
SDK comes in form of installable shell script: shar
(shell archive).
\end{block}
\begin{lstlisting}[style=Console]
$ ./***-toolchain-***.sh
Enter target directory for SDK (default: /opt/***/1.6.1): sdk
You are about to install the SDK to "/***/sdk". Proceed[Y/n]?
Extracting SDK...done
Setting it up...done
SDK has been successfully set up and is ready to be used.
\end{lstlisting}
\end{frame}

\begin{frame}[fragile]{SDK has a directory structure}
\begin{block}{}
\begin{itemize}
\item{setup script}
\item{target sysroot}
\item{toolchain sysroot}
\end{itemize}
\end{block}
\begin{lstlisting}[style=TinyConsole]
sdk
|-- environment-setup-core2-64-nokia-linux
|-- site-config-core2-64-nokia-linux
|-- sysroots
|   |-- core2-64-nokia-linux
|   |   |-- etc
|   |   |-- lib64
|   |   |-- usr
|   |   \-- var
|   \-- i686-nsnsdk-linux
|       |-- bin
|       |-- etc
|       |-- lib
|       |-- usr
|       |-- var
|       \-- x86_64-pc-linux-gnu
\-- version-core2-64-nokia-linux
\end{lstlisting}
\end{frame}

\begin{frame}[fragile, t]{SDK has environment script}
    \begin{block}{}
        Sets path, auxiliary and program specific variables:
        \begin{itemize}
            \scriptsize
            \item<1->{to know location of sysroots}
            \item<2->{to use \verb|[archprefix]-gcc|, \verb|cmake|, \verb|opkg|}
            \item<3->{for \verb|pkg-config|, \verb|opkg|, \verb|python| to use sdk}
            \item<4->{\verb|CC|, \verb|CXX|, \verb|LD| for build tool to detect}
            \item<5->{libdir and dynlinker to execute binaries}
        \end{itemize}
    \end{block}
    \begin{onlyenv}<1>
\begin{lstlisting}[style=TinyConsole]
export SDKTARGETSYSROOT=/***/sdk/sysroots/core2-64-nokia-linux
export OECORE_NATIVE_SYSROOT="/***/sdk/sysroots/i686-nsnsdk-linux"
\end{lstlisting}
    \end{onlyenv}
    \begin{onlyenv}<2>
\begin{lstlisting}[style=TinyConsole]
export PATH=/***/sdk/sysroots/i686-nsnsdk-linux/usr/bin: \
            /***/sdk/sysroots/i686-nsnsdk-linux/usr/bin/x86_64-nokia-linux: \
            $PATH
\end{lstlisting}
    \end{onlyenv}
    \begin{onlyenv}<3>
\begin{lstlisting}[style=TinyConsole]
export CC="x86_64-pc-linux-gnu-gcc -m64 --sysroot=$SDKTARGETSYSROOT"
export CXX="x86_64-pc-linux-gnu-g++ -m64 --sysroot=$SDKTARGETSYSROOT"
export CPP="x86_64-pc-linux-gnu-gcc -E -m64 --sysroot=$SDKTARGETSYSROOT"
export AS="x86_64-pc-linux-gnu-as"
export LD="x86_64-pc-linux-gnu-ld --sysroot=$SDKTARGETSYSROOT"
\end{lstlisting}
    \end{onlyenv}
    \begin{onlyenv}<4>
\begin{lstlisting}[style=TinyConsole]
export PKG_CONFIG_SYSROOT_DIR=$SDKTARGETSYSROOT
export PKG_CONFIG_PATH=$SDKTARGETSYSROOT/usr/lib64/pkgconfig
export OFFLINE_ROOT=$SDKTARGETSYSROOT
export PYTHONPATH=$OECORE_NATIVE_SYSROOT/usr/lib/python2.7/site-packages:$PYTHONPATH
\end{lstlisting}
    \end{onlyenv}
    \begin{onlyenv}<5>
\begin{lstlisting}[style=TinyConsole]
export SYSROOT_LIB_DIR=$SDKTARGETSYSROOT/usr/lib64:$SDKTARGETSYSROOT/lib64
export SYSROOT_DYN_LINKER=$SDKTARGETSYSROOT/lib64/ld-linux-x86-64.so.2
\end{lstlisting}
    \end{onlyenv}
\end{frame}

\subsection{Package}

\begin{frame}
\end{frame}

\subsection{Index}

\begin{frame}
\end{frame}

\section{Roles}

\subsection{Maintainer}

\begin{frame}
\end{frame}

\subsection{Developer}

\begin{frame}
\end{frame}

\end{document}
